\documentclass[11pt]{article}

    \usepackage[breakable]{tcolorbox}
    \usepackage{parskip} % Stop auto-indenting (to mimic markdown behaviour)
    
    \usepackage{iftex}
    \ifPDFTeX
    	\usepackage[T1]{fontenc}
    	\usepackage{mathpazo}
    \else
    	\usepackage{fontspec}
    \fi

    % Basic figure setup, for now with no caption control since it's done
    % automatically by Pandoc (which extracts ![](path) syntax from Markdown).
    \usepackage{graphicx}
    % Maintain compatibility with old templates. Remove in nbconvert 6.0
    \let\Oldincludegraphics\includegraphics
    % Ensure that by default, figures have no caption (until we provide a
    % proper Figure object with a Caption API and a way to capture that
    % in the conversion process - todo).
    \usepackage{caption}
    \DeclareCaptionFormat{nocaption}{}
    \captionsetup{format=nocaption,aboveskip=0pt,belowskip=0pt}

    \usepackage[Export]{adjustbox} % Used to constrain images to a maximum size
    \adjustboxset{max size={0.9\linewidth}{0.9\paperheight}}
    \usepackage{float}
    \floatplacement{figure}{H} % forces figures to be placed at the correct location
    \usepackage{xcolor} % Allow colors to be defined
    \usepackage{enumerate} % Needed for markdown enumerations to work
    \usepackage{geometry} % Used to adjust the document margins
    \usepackage{amsmath} % Equations
    \usepackage{amssymb} % Equations
    \usepackage{textcomp} % defines textquotesingle
    % Hack from http://tex.stackexchange.com/a/47451/13684:
    \AtBeginDocument{%
        \def\PYZsq{\textquotesingle}% Upright quotes in Pygmentized code
    }
    \usepackage{upquote} % Upright quotes for verbatim code
    \usepackage{eurosym} % defines \euro
    \usepackage[mathletters]{ucs} % Extended unicode (utf-8) support
    \usepackage{fancyvrb} % verbatim replacement that allows latex
    \usepackage{grffile} % extends the file name processing of package graphics 
                         % to support a larger range
    \makeatletter % fix for grffile with XeLaTeX
    \def\Gread@@xetex#1{%
      \IfFileExists{"\Gin@base".bb}%
      {\Gread@eps{\Gin@base.bb}}%
      {\Gread@@xetex@aux#1}%
    }
    \makeatother

    % The hyperref package gives us a pdf with properly built
    % internal navigation ('pdf bookmarks' for the table of contents,
    % internal cross-reference links, web links for URLs, etc.)
    \usepackage{hyperref}
    % The default LaTeX title has an obnoxious amount of whitespace. By default,
    % titling removes some of it. It also provides customization options.
    \usepackage{titling}
    \usepackage{longtable} % longtable support required by pandoc >1.10
    \usepackage{booktabs}  % table support for pandoc > 1.12.2
    \usepackage[inline]{enumitem} % IRkernel/repr support (it uses the enumerate* environment)
    \usepackage[normalem]{ulem} % ulem is needed to support strikethroughs (\sout)
                                % normalem makes italics be italics, not underlines
    \usepackage{mathrsfs}
    

    
    % Colors for the hyperref package
    \definecolor{urlcolor}{rgb}{0,.145,.698}
    \definecolor{linkcolor}{rgb}{.71,0.21,0.01}
    \definecolor{citecolor}{rgb}{.12,.54,.11}

    % ANSI colors
    \definecolor{ansi-black}{HTML}{3E424D}
    \definecolor{ansi-black-intense}{HTML}{282C36}
    \definecolor{ansi-red}{HTML}{E75C58}
    \definecolor{ansi-red-intense}{HTML}{B22B31}
    \definecolor{ansi-green}{HTML}{00A250}
    \definecolor{ansi-green-intense}{HTML}{007427}
    \definecolor{ansi-yellow}{HTML}{DDB62B}
    \definecolor{ansi-yellow-intense}{HTML}{B27D12}
    \definecolor{ansi-blue}{HTML}{208FFB}
    \definecolor{ansi-blue-intense}{HTML}{0065CA}
    \definecolor{ansi-magenta}{HTML}{D160C4}
    \definecolor{ansi-magenta-intense}{HTML}{A03196}
    \definecolor{ansi-cyan}{HTML}{60C6C8}
    \definecolor{ansi-cyan-intense}{HTML}{258F8F}
    \definecolor{ansi-white}{HTML}{C5C1B4}
    \definecolor{ansi-white-intense}{HTML}{A1A6B2}
    \definecolor{ansi-default-inverse-fg}{HTML}{FFFFFF}
    \definecolor{ansi-default-inverse-bg}{HTML}{000000}

    % commands and environments needed by pandoc snippets
    % extracted from the output of `pandoc -s`
    \providecommand{\tightlist}{%
      \setlength{\itemsep}{0pt}\setlength{\parskip}{0pt}}
    \DefineVerbatimEnvironment{Highlighting}{Verbatim}{commandchars=\\\{\}}
    % Add ',fontsize=\small' for more characters per line
    \newenvironment{Shaded}{}{}
    \newcommand{\KeywordTok}[1]{\textcolor[rgb]{0.00,0.44,0.13}{\textbf{{#1}}}}
    \newcommand{\DataTypeTok}[1]{\textcolor[rgb]{0.56,0.13,0.00}{{#1}}}
    \newcommand{\DecValTok}[1]{\textcolor[rgb]{0.25,0.63,0.44}{{#1}}}
    \newcommand{\BaseNTok}[1]{\textcolor[rgb]{0.25,0.63,0.44}{{#1}}}
    \newcommand{\FloatTok}[1]{\textcolor[rgb]{0.25,0.63,0.44}{{#1}}}
    \newcommand{\CharTok}[1]{\textcolor[rgb]{0.25,0.44,0.63}{{#1}}}
    \newcommand{\StringTok}[1]{\textcolor[rgb]{0.25,0.44,0.63}{{#1}}}
    \newcommand{\CommentTok}[1]{\textcolor[rgb]{0.38,0.63,0.69}{\textit{{#1}}}}
    \newcommand{\OtherTok}[1]{\textcolor[rgb]{0.00,0.44,0.13}{{#1}}}
    \newcommand{\AlertTok}[1]{\textcolor[rgb]{1.00,0.00,0.00}{\textbf{{#1}}}}
    \newcommand{\FunctionTok}[1]{\textcolor[rgb]{0.02,0.16,0.49}{{#1}}}
    \newcommand{\RegionMarkerTok}[1]{{#1}}
    \newcommand{\ErrorTok}[1]{\textcolor[rgb]{1.00,0.00,0.00}{\textbf{{#1}}}}
    \newcommand{\NormalTok}[1]{{#1}}
    
    % Additional commands for more recent versions of Pandoc
    \newcommand{\ConstantTok}[1]{\textcolor[rgb]{0.53,0.00,0.00}{{#1}}}
    \newcommand{\SpecialCharTok}[1]{\textcolor[rgb]{0.25,0.44,0.63}{{#1}}}
    \newcommand{\VerbatimStringTok}[1]{\textcolor[rgb]{0.25,0.44,0.63}{{#1}}}
    \newcommand{\SpecialStringTok}[1]{\textcolor[rgb]{0.73,0.40,0.53}{{#1}}}
    \newcommand{\ImportTok}[1]{{#1}}
    \newcommand{\DocumentationTok}[1]{\textcolor[rgb]{0.73,0.13,0.13}{\textit{{#1}}}}
    \newcommand{\AnnotationTok}[1]{\textcolor[rgb]{0.38,0.63,0.69}{\textbf{\textit{{#1}}}}}
    \newcommand{\CommentVarTok}[1]{\textcolor[rgb]{0.38,0.63,0.69}{\textbf{\textit{{#1}}}}}
    \newcommand{\VariableTok}[1]{\textcolor[rgb]{0.10,0.09,0.49}{{#1}}}
    \newcommand{\ControlFlowTok}[1]{\textcolor[rgb]{0.00,0.44,0.13}{\textbf{{#1}}}}
    \newcommand{\OperatorTok}[1]{\textcolor[rgb]{0.40,0.40,0.40}{{#1}}}
    \newcommand{\BuiltInTok}[1]{{#1}}
    \newcommand{\ExtensionTok}[1]{{#1}}
    \newcommand{\PreprocessorTok}[1]{\textcolor[rgb]{0.74,0.48,0.00}{{#1}}}
    \newcommand{\AttributeTok}[1]{\textcolor[rgb]{0.49,0.56,0.16}{{#1}}}
    \newcommand{\InformationTok}[1]{\textcolor[rgb]{0.38,0.63,0.69}{\textbf{\textit{{#1}}}}}
    \newcommand{\WarningTok}[1]{\textcolor[rgb]{0.38,0.63,0.69}{\textbf{\textit{{#1}}}}}
    
    
    % Define a nice break command that doesn't care if a line doesn't already
    % exist.
    \def\br{\hspace*{\fill} \\* }
    % Math Jax compatibility definitions
    \def\gt{>}
    \def\lt{<}
    \let\Oldtex\TeX
    \let\Oldlatex\LaTeX
    \renewcommand{\TeX}{\textrm{\Oldtex}}
    \renewcommand{\LaTeX}{\textrm{\Oldlatex}}
    % Document parameters
    % Document title
    \title{QOSF Screening Task 1 Solution\\
    \vspace{1ex}
    \small{by Cenk T\"uys\"uz}
    }
    
    
    
    
    
    
% Pygments definitions
\makeatletter
\def\PY@reset{\let\PY@it=\relax \let\PY@bf=\relax%
    \let\PY@ul=\relax \let\PY@tc=\relax%
    \let\PY@bc=\relax \let\PY@ff=\relax}
\def\PY@tok#1{\csname PY@tok@#1\endcsname}
\def\PY@toks#1+{\ifx\relax#1\empty\else%
    \PY@tok{#1}\expandafter\PY@toks\fi}
\def\PY@do#1{\PY@bc{\PY@tc{\PY@ul{%
    \PY@it{\PY@bf{\PY@ff{#1}}}}}}}
\def\PY#1#2{\PY@reset\PY@toks#1+\relax+\PY@do{#2}}

\expandafter\def\csname PY@tok@w\endcsname{\def\PY@tc##1{\textcolor[rgb]{0.73,0.73,0.73}{##1}}}
\expandafter\def\csname PY@tok@c\endcsname{\let\PY@it=\textit\def\PY@tc##1{\textcolor[rgb]{0.25,0.50,0.50}{##1}}}
\expandafter\def\csname PY@tok@cp\endcsname{\def\PY@tc##1{\textcolor[rgb]{0.74,0.48,0.00}{##1}}}
\expandafter\def\csname PY@tok@k\endcsname{\let\PY@bf=\textbf\def\PY@tc##1{\textcolor[rgb]{0.00,0.50,0.00}{##1}}}
\expandafter\def\csname PY@tok@kp\endcsname{\def\PY@tc##1{\textcolor[rgb]{0.00,0.50,0.00}{##1}}}
\expandafter\def\csname PY@tok@kt\endcsname{\def\PY@tc##1{\textcolor[rgb]{0.69,0.00,0.25}{##1}}}
\expandafter\def\csname PY@tok@o\endcsname{\def\PY@tc##1{\textcolor[rgb]{0.40,0.40,0.40}{##1}}}
\expandafter\def\csname PY@tok@ow\endcsname{\let\PY@bf=\textbf\def\PY@tc##1{\textcolor[rgb]{0.67,0.13,1.00}{##1}}}
\expandafter\def\csname PY@tok@nb\endcsname{\def\PY@tc##1{\textcolor[rgb]{0.00,0.50,0.00}{##1}}}
\expandafter\def\csname PY@tok@nf\endcsname{\def\PY@tc##1{\textcolor[rgb]{0.00,0.00,1.00}{##1}}}
\expandafter\def\csname PY@tok@nc\endcsname{\let\PY@bf=\textbf\def\PY@tc##1{\textcolor[rgb]{0.00,0.00,1.00}{##1}}}
\expandafter\def\csname PY@tok@nn\endcsname{\let\PY@bf=\textbf\def\PY@tc##1{\textcolor[rgb]{0.00,0.00,1.00}{##1}}}
\expandafter\def\csname PY@tok@ne\endcsname{\let\PY@bf=\textbf\def\PY@tc##1{\textcolor[rgb]{0.82,0.25,0.23}{##1}}}
\expandafter\def\csname PY@tok@nv\endcsname{\def\PY@tc##1{\textcolor[rgb]{0.10,0.09,0.49}{##1}}}
\expandafter\def\csname PY@tok@no\endcsname{\def\PY@tc##1{\textcolor[rgb]{0.53,0.00,0.00}{##1}}}
\expandafter\def\csname PY@tok@nl\endcsname{\def\PY@tc##1{\textcolor[rgb]{0.63,0.63,0.00}{##1}}}
\expandafter\def\csname PY@tok@ni\endcsname{\let\PY@bf=\textbf\def\PY@tc##1{\textcolor[rgb]{0.60,0.60,0.60}{##1}}}
\expandafter\def\csname PY@tok@na\endcsname{\def\PY@tc##1{\textcolor[rgb]{0.49,0.56,0.16}{##1}}}
\expandafter\def\csname PY@tok@nt\endcsname{\let\PY@bf=\textbf\def\PY@tc##1{\textcolor[rgb]{0.00,0.50,0.00}{##1}}}
\expandafter\def\csname PY@tok@nd\endcsname{\def\PY@tc##1{\textcolor[rgb]{0.67,0.13,1.00}{##1}}}
\expandafter\def\csname PY@tok@s\endcsname{\def\PY@tc##1{\textcolor[rgb]{0.73,0.13,0.13}{##1}}}
\expandafter\def\csname PY@tok@sd\endcsname{\let\PY@it=\textit\def\PY@tc##1{\textcolor[rgb]{0.73,0.13,0.13}{##1}}}
\expandafter\def\csname PY@tok@si\endcsname{\let\PY@bf=\textbf\def\PY@tc##1{\textcolor[rgb]{0.73,0.40,0.53}{##1}}}
\expandafter\def\csname PY@tok@se\endcsname{\let\PY@bf=\textbf\def\PY@tc##1{\textcolor[rgb]{0.73,0.40,0.13}{##1}}}
\expandafter\def\csname PY@tok@sr\endcsname{\def\PY@tc##1{\textcolor[rgb]{0.73,0.40,0.53}{##1}}}
\expandafter\def\csname PY@tok@ss\endcsname{\def\PY@tc##1{\textcolor[rgb]{0.10,0.09,0.49}{##1}}}
\expandafter\def\csname PY@tok@sx\endcsname{\def\PY@tc##1{\textcolor[rgb]{0.00,0.50,0.00}{##1}}}
\expandafter\def\csname PY@tok@m\endcsname{\def\PY@tc##1{\textcolor[rgb]{0.40,0.40,0.40}{##1}}}
\expandafter\def\csname PY@tok@gh\endcsname{\let\PY@bf=\textbf\def\PY@tc##1{\textcolor[rgb]{0.00,0.00,0.50}{##1}}}
\expandafter\def\csname PY@tok@gu\endcsname{\let\PY@bf=\textbf\def\PY@tc##1{\textcolor[rgb]{0.50,0.00,0.50}{##1}}}
\expandafter\def\csname PY@tok@gd\endcsname{\def\PY@tc##1{\textcolor[rgb]{0.63,0.00,0.00}{##1}}}
\expandafter\def\csname PY@tok@gi\endcsname{\def\PY@tc##1{\textcolor[rgb]{0.00,0.63,0.00}{##1}}}
\expandafter\def\csname PY@tok@gr\endcsname{\def\PY@tc##1{\textcolor[rgb]{1.00,0.00,0.00}{##1}}}
\expandafter\def\csname PY@tok@ge\endcsname{\let\PY@it=\textit}
\expandafter\def\csname PY@tok@gs\endcsname{\let\PY@bf=\textbf}
\expandafter\def\csname PY@tok@gp\endcsname{\let\PY@bf=\textbf\def\PY@tc##1{\textcolor[rgb]{0.00,0.00,0.50}{##1}}}
\expandafter\def\csname PY@tok@go\endcsname{\def\PY@tc##1{\textcolor[rgb]{0.53,0.53,0.53}{##1}}}
\expandafter\def\csname PY@tok@gt\endcsname{\def\PY@tc##1{\textcolor[rgb]{0.00,0.27,0.87}{##1}}}
\expandafter\def\csname PY@tok@err\endcsname{\def\PY@bc##1{\setlength{\fboxsep}{0pt}\fcolorbox[rgb]{1.00,0.00,0.00}{1,1,1}{\strut ##1}}}
\expandafter\def\csname PY@tok@kc\endcsname{\let\PY@bf=\textbf\def\PY@tc##1{\textcolor[rgb]{0.00,0.50,0.00}{##1}}}
\expandafter\def\csname PY@tok@kd\endcsname{\let\PY@bf=\textbf\def\PY@tc##1{\textcolor[rgb]{0.00,0.50,0.00}{##1}}}
\expandafter\def\csname PY@tok@kn\endcsname{\let\PY@bf=\textbf\def\PY@tc##1{\textcolor[rgb]{0.00,0.50,0.00}{##1}}}
\expandafter\def\csname PY@tok@kr\endcsname{\let\PY@bf=\textbf\def\PY@tc##1{\textcolor[rgb]{0.00,0.50,0.00}{##1}}}
\expandafter\def\csname PY@tok@bp\endcsname{\def\PY@tc##1{\textcolor[rgb]{0.00,0.50,0.00}{##1}}}
\expandafter\def\csname PY@tok@fm\endcsname{\def\PY@tc##1{\textcolor[rgb]{0.00,0.00,1.00}{##1}}}
\expandafter\def\csname PY@tok@vc\endcsname{\def\PY@tc##1{\textcolor[rgb]{0.10,0.09,0.49}{##1}}}
\expandafter\def\csname PY@tok@vg\endcsname{\def\PY@tc##1{\textcolor[rgb]{0.10,0.09,0.49}{##1}}}
\expandafter\def\csname PY@tok@vi\endcsname{\def\PY@tc##1{\textcolor[rgb]{0.10,0.09,0.49}{##1}}}
\expandafter\def\csname PY@tok@vm\endcsname{\def\PY@tc##1{\textcolor[rgb]{0.10,0.09,0.49}{##1}}}
\expandafter\def\csname PY@tok@sa\endcsname{\def\PY@tc##1{\textcolor[rgb]{0.73,0.13,0.13}{##1}}}
\expandafter\def\csname PY@tok@sb\endcsname{\def\PY@tc##1{\textcolor[rgb]{0.73,0.13,0.13}{##1}}}
\expandafter\def\csname PY@tok@sc\endcsname{\def\PY@tc##1{\textcolor[rgb]{0.73,0.13,0.13}{##1}}}
\expandafter\def\csname PY@tok@dl\endcsname{\def\PY@tc##1{\textcolor[rgb]{0.73,0.13,0.13}{##1}}}
\expandafter\def\csname PY@tok@s2\endcsname{\def\PY@tc##1{\textcolor[rgb]{0.73,0.13,0.13}{##1}}}
\expandafter\def\csname PY@tok@sh\endcsname{\def\PY@tc##1{\textcolor[rgb]{0.73,0.13,0.13}{##1}}}
\expandafter\def\csname PY@tok@s1\endcsname{\def\PY@tc##1{\textcolor[rgb]{0.73,0.13,0.13}{##1}}}
\expandafter\def\csname PY@tok@mb\endcsname{\def\PY@tc##1{\textcolor[rgb]{0.40,0.40,0.40}{##1}}}
\expandafter\def\csname PY@tok@mf\endcsname{\def\PY@tc##1{\textcolor[rgb]{0.40,0.40,0.40}{##1}}}
\expandafter\def\csname PY@tok@mh\endcsname{\def\PY@tc##1{\textcolor[rgb]{0.40,0.40,0.40}{##1}}}
\expandafter\def\csname PY@tok@mi\endcsname{\def\PY@tc##1{\textcolor[rgb]{0.40,0.40,0.40}{##1}}}
\expandafter\def\csname PY@tok@il\endcsname{\def\PY@tc##1{\textcolor[rgb]{0.40,0.40,0.40}{##1}}}
\expandafter\def\csname PY@tok@mo\endcsname{\def\PY@tc##1{\textcolor[rgb]{0.40,0.40,0.40}{##1}}}
\expandafter\def\csname PY@tok@ch\endcsname{\let\PY@it=\textit\def\PY@tc##1{\textcolor[rgb]{0.25,0.50,0.50}{##1}}}
\expandafter\def\csname PY@tok@cm\endcsname{\let\PY@it=\textit\def\PY@tc##1{\textcolor[rgb]{0.25,0.50,0.50}{##1}}}
\expandafter\def\csname PY@tok@cpf\endcsname{\let\PY@it=\textit\def\PY@tc##1{\textcolor[rgb]{0.25,0.50,0.50}{##1}}}
\expandafter\def\csname PY@tok@c1\endcsname{\let\PY@it=\textit\def\PY@tc##1{\textcolor[rgb]{0.25,0.50,0.50}{##1}}}
\expandafter\def\csname PY@tok@cs\endcsname{\let\PY@it=\textit\def\PY@tc##1{\textcolor[rgb]{0.25,0.50,0.50}{##1}}}

\def\PYZbs{\char`\\}
\def\PYZus{\char`\_}
\def\PYZob{\char`\{}
\def\PYZcb{\char`\}}
\def\PYZca{\char`\^}
\def\PYZam{\char`\&}
\def\PYZlt{\char`\<}
\def\PYZgt{\char`\>}
\def\PYZsh{\char`\#}
\def\PYZpc{\char`\%}
\def\PYZdl{\char`\$}
\def\PYZhy{\char`\-}
\def\PYZsq{\char`\'}
\def\PYZdq{\char`\"}
\def\PYZti{\char`\~}
% for compatibility with earlier versions
\def\PYZat{@}
\def\PYZlb{[}
\def\PYZrb{]}
\makeatother


    % For linebreaks inside Verbatim environment from package fancyvrb. 
    \makeatletter
        \newbox\Wrappedcontinuationbox 
        \newbox\Wrappedvisiblespacebox 
        \newcommand*\Wrappedvisiblespace {\textcolor{red}{\textvisiblespace}} 
        \newcommand*\Wrappedcontinuationsymbol {\textcolor{red}{\llap{\tiny$\m@th\hookrightarrow$}}} 
        \newcommand*\Wrappedcontinuationindent {3ex } 
        \newcommand*\Wrappedafterbreak {\kern\Wrappedcontinuationindent\copy\Wrappedcontinuationbox} 
        % Take advantage of the already applied Pygments mark-up to insert 
        % potential linebreaks for TeX processing. 
        %        {, <, #, %, $, ' and ": go to next line. 
        %        _, }, ^, &, >, - and ~: stay at end of broken line. 
        % Use of \textquotesingle for straight quote. 
        \newcommand*\Wrappedbreaksatspecials {% 
            \def\PYGZus{\discretionary{\char`\_}{\Wrappedafterbreak}{\char`\_}}% 
            \def\PYGZob{\discretionary{}{\Wrappedafterbreak\char`\{}{\char`\{}}% 
            \def\PYGZcb{\discretionary{\char`\}}{\Wrappedafterbreak}{\char`\}}}% 
            \def\PYGZca{\discretionary{\char`\^}{\Wrappedafterbreak}{\char`\^}}% 
            \def\PYGZam{\discretionary{\char`\&}{\Wrappedafterbreak}{\char`\&}}% 
            \def\PYGZlt{\discretionary{}{\Wrappedafterbreak\char`\<}{\char`\<}}% 
            \def\PYGZgt{\discretionary{\char`\>}{\Wrappedafterbreak}{\char`\>}}% 
            \def\PYGZsh{\discretionary{}{\Wrappedafterbreak\char`\#}{\char`\#}}% 
            \def\PYGZpc{\discretionary{}{\Wrappedafterbreak\char`\%}{\char`\%}}% 
            \def\PYGZdl{\discretionary{}{\Wrappedafterbreak\char`\$}{\char`\$}}% 
            \def\PYGZhy{\discretionary{\char`\-}{\Wrappedafterbreak}{\char`\-}}% 
            \def\PYGZsq{\discretionary{}{\Wrappedafterbreak\textquotesingle}{\textquotesingle}}% 
            \def\PYGZdq{\discretionary{}{\Wrappedafterbreak\char`\"}{\char`\"}}% 
            \def\PYGZti{\discretionary{\char`\~}{\Wrappedafterbreak}{\char`\~}}% 
        } 
        % Some characters . , ; ? ! / are not pygmentized. 
        % This macro makes them "active" and they will insert potential linebreaks 
        \newcommand*\Wrappedbreaksatpunct {% 
            \lccode`\~`\.\lowercase{\def~}{\discretionary{\hbox{\char`\.}}{\Wrappedafterbreak}{\hbox{\char`\.}}}% 
            \lccode`\~`\,\lowercase{\def~}{\discretionary{\hbox{\char`\,}}{\Wrappedafterbreak}{\hbox{\char`\,}}}% 
            \lccode`\~`\;\lowercase{\def~}{\discretionary{\hbox{\char`\;}}{\Wrappedafterbreak}{\hbox{\char`\;}}}% 
            \lccode`\~`\:\lowercase{\def~}{\discretionary{\hbox{\char`\:}}{\Wrappedafterbreak}{\hbox{\char`\:}}}% 
            \lccode`\~`\?\lowercase{\def~}{\discretionary{\hbox{\char`\?}}{\Wrappedafterbreak}{\hbox{\char`\?}}}% 
            \lccode`\~`\!\lowercase{\def~}{\discretionary{\hbox{\char`\!}}{\Wrappedafterbreak}{\hbox{\char`\!}}}% 
            \lccode`\~`\/\lowercase{\def~}{\discretionary{\hbox{\char`\/}}{\Wrappedafterbreak}{\hbox{\char`\/}}}% 
            \catcode`\.\active
            \catcode`\,\active 
            \catcode`\;\active
            \catcode`\:\active
            \catcode`\?\active
            \catcode`\!\active
            \catcode`\/\active 
            \lccode`\~`\~ 	
        }
    \makeatother

    \let\OriginalVerbatim=\Verbatim
    \makeatletter
    \renewcommand{\Verbatim}[1][1]{%
        %\parskip\z@skip
        \sbox\Wrappedcontinuationbox {\Wrappedcontinuationsymbol}%
        \sbox\Wrappedvisiblespacebox {\FV@SetupFont\Wrappedvisiblespace}%
        \def\FancyVerbFormatLine ##1{\hsize\linewidth
            \vtop{\raggedright\hyphenpenalty\z@\exhyphenpenalty\z@
                \doublehyphendemerits\z@\finalhyphendemerits\z@
                \strut ##1\strut}%
        }%
        % If the linebreak is at a space, the latter will be displayed as visible
        % space at end of first line, and a continuation symbol starts next line.
        % Stretch/shrink are however usually zero for typewriter font.
        \def\FV@Space {%
            \nobreak\hskip\z@ plus\fontdimen3\font minus\fontdimen4\font
            \discretionary{\copy\Wrappedvisiblespacebox}{\Wrappedafterbreak}
            {\kern\fontdimen2\font}%
        }%
        
        % Allow breaks at special characters using \PYG... macros.
        \Wrappedbreaksatspecials
        % Breaks at punctuation characters . , ; ? ! and / need catcode=\active 	
        \OriginalVerbatim[#1,codes*=\Wrappedbreaksatpunct]%
    }
    \makeatother

    % Exact colors from NB
    \definecolor{incolor}{HTML}{303F9F}
    \definecolor{outcolor}{HTML}{D84315}
    \definecolor{cellborder}{HTML}{CFCFCF}
    \definecolor{cellbackground}{HTML}{F7F7F7}
    
    % prompt
    \makeatletter
    \newcommand{\boxspacing}{\kern\kvtcb@left@rule\kern\kvtcb@boxsep}
    \makeatother
    \newcommand{\prompt}[4]{
        \ttfamily\llap{{\color{#2}[#3]:\hspace{3pt}#4}}\vspace{-\baselineskip}
    }
    

    
    % Prevent overflowing lines due to hard-to-break entities
    \sloppy 
    % Setup hyperref package
    \hypersetup{
      breaklinks=true,  % so long urls are correctly broken across lines
      colorlinks=true,
      urlcolor=urlcolor,
      linkcolor=linkcolor,
      citecolor=citecolor,
      }
    % Slightly bigger margins than the latex defaults
    
    \geometry{verbose,tmargin=1in,bmargin=1in,lmargin=1in,rmargin=1in}
    
    

\begin{document}
    
    \maketitle
    
    

    
    \hypertarget{task1}{%
\section{Task1}\label{task1}}

\[\newcommand{\ket}[1]{\left|{#1}\right\rangle}\] Implement, on a
quantum simulator of your choice, the following 4 qubits state
\(\newcommand{\ket}[1]{\left|{#1}\right\rangle} \ket{\psi(\theta)}\):

    \begin{center}
        \adjustimage{width=0.6\linewidth}{../../images/circuit_ansatz-1.jpg}

    \end{center}

Where the number of layers, denoted with L, has to be considered as a
parameter. We call ``Layer'' the combination of 1 yellow + 1 green
block, so, for example, U1 + U2 is a layer. The odd/even variational
blocks are given by:

Even blocks:

    \begin{center}
        \adjustimage{width=0.6\linewidth}{../../images/even_block.jpg}

    \end{center}

Odd blocks:

    \begin{center}
        \adjustimage{width=0.6\linewidth}{../../images/odd_block.jpg}

    \end{center}

The angles \(\theta_{i,n}\) are variational parameters, lying in the
interval \((0, 2\pi)\), initialized at random. Double qubit gates are CZ
gates.

Report with a plot, as a function of the number of layers, L, the
minimum distance

\[\newcommand{\ket}[1]{\left|{#1}\right\rangle} \varepsilon = min_\theta \parallel \ket{\psi(\theta)} - \ket{\phi} \parallel  \]

Where \(\newcommand{\ket}[1]{\left|{#1}\right\rangle} \ket{\phi}\) is a
randomly generated vector on 4 qubits and the norm
\(\newcommand{\ket}[1]{\left|{#1}\right\rangle} \parallel \ket{v} \parallel\),
of a state \(\newcommand{\ket}[1]{\left|{#1}\right\rangle} \ket{v}\),
simply denotes the sum of the squares of the components of
\(\newcommand{\ket}[1]{\left|{#1}\right\rangle} \ket{v}\). The right set
of parameters \(\theta_{i,n}\) can be found via any method of choice
(e.g.~grid-search or gradient descent)

\hypertarget{bonus-question}{%
\subsection{Bonus question:}\label{bonus-question}}

Try using other gates for the parametrized gates and see what happens.

    \hypertarget{solution}{%
\section{Solution}\label{solution}}

Let's start by importing NumPy, Qiskit and other relevant libraries.

    \begin{tcolorbox}[breakable, size=fbox, boxrule=1pt, pad at break*=1mm,colback=cellbackground, colframe=cellborder]
\prompt{In}{incolor}{1}{\boxspacing}
\begin{Verbatim}[commandchars=\\\{\}]
\PY{k+kn}{import} \PY{n+nn}{numpy} \PY{k}{as} \PY{n+nn}{np}
\PY{k+kn}{from} \PY{n+nn}{qiskit} \PY{k+kn}{import} \PY{n}{Aer}\PY{p}{,} \PY{n}{QuantumRegister}\PY{p}{,} \PY{n}{QuantumCircuit}\PY{p}{,} \PY{n}{execute}
\PY{k+kn}{from} \PY{n+nn}{qiskit}\PY{n+nn}{.}\PY{n+nn}{quantum\PYZus{}info} \PY{k+kn}{import} \PY{n}{random\PYZus{}statevector}
\PY{k+kn}{from} \PY{n+nn}{scipy}\PY{n+nn}{.}\PY{n+nn}{optimize} \PY{k+kn}{import} \PY{n}{minimize}

\PY{o}{\PYZpc{}}\PY{k}{matplotlib} inline
\PY{k+kn}{import} \PY{n+nn}{matplotlib}\PY{n+nn}{.}\PY{n+nn}{pyplot} \PY{k}{as} \PY{n+nn}{plt}
\PY{k+kn}{import} \PY{n+nn}{matplotlib} \PY{k}{as} \PY{n+nn}{mpl}
\PY{n}{mpl}\PY{o}{.}\PY{n}{rcParams}\PY{p}{[}\PY{l+s+s1}{\PYZsq{}}\PY{l+s+s1}{figure.dpi}\PY{l+s+s1}{\PYZsq{}}\PY{p}{]}\PY{o}{=} \PY{l+m+mi}{200}

\PY{k+kn}{import} \PY{n+nn}{time}
\end{Verbatim}
\end{tcolorbox}

    Now, we can define the Parametrized Quantum Circuits. We construct a QNN
class to build our model under a single object. The QNN class contains
several functions that we need. Let's go over them quickly. (Read the
comments in the code for more details).
\begin{itemize}
 \item \colorbox{lightgray}{\_\_init\_\_()} is
called when the QNN() is first called. It creates the random set of
parameters, as well as the random Haar state. A fixed seed is chosen for
the random state, such that results are consistent for each execution of
the program. 
 \item \colorbox{lightgray}{qc()} takes a set of parameters as input and
applies the Quantum Circuit defined in the question. 
\item  \colorbox{lightgray}{gate()}
takes the gate name as a string and applies that gate to the specified qubit.
\item  \colorbox{lightgray}{cost\_fn()} calculates the \(\varepsilon\) defined in
the question w.r.t. a random Haar state. 
\item  \colorbox{lightgray}{train()} uses the
COBYLA optimizer to minimize the cost and find optimal parameters and
returns the final \(\varepsilon\).
\end{itemize}
\newpage

    \begin{tcolorbox}[breakable, size=fbox, boxrule=1pt, pad at break*=1mm,colback=cellbackground, colframe=cellborder]
\prompt{In}{incolor}{2}{\boxspacing}
\begin{Verbatim}[commandchars=\\\{\}]
\PY{k}{class} \PY{n+nc}{QNN}\PY{p}{(}\PY{p}{)}\PY{p}{:}
    \PY{l+s+sd}{\PYZsq{}\PYZsq{}\PYZsq{} Constructs the Quantum Neural Network (QNN) object.}
\PY{l+s+sd}{    Args:}
\PY{l+s+sd}{    =====}
\PY{l+s+sd}{    n\PYZus{}layers: int}
\PY{l+s+sd}{        Number of layers, which the even and odd blocks are repeated.}
\PY{l+s+sd}{    steps: int}
\PY{l+s+sd}{        Number of optimization steps}
\PY{l+s+sd}{    lr: float}
\PY{l+s+sd}{        Learning rate}
\PY{l+s+sd}{    \PYZsq{}\PYZsq{}\PYZsq{}}
    \PY{k}{def} \PY{n+nf+fm}{\PYZus{}\PYZus{}init\PYZus{}\PYZus{}}\PY{p}{(}\PY{n+nb+bp}{self}\PY{p}{,}\PY{n}{n\PYZus{}layers}\PY{o}{=}\PY{l+m+mi}{1}\PY{p}{,} \PY{n}{gate\PYZus{}set}\PY{o}{=}\PY{p}{[}\PY{l+s+s1}{\PYZsq{}}\PY{l+s+s1}{rz}\PY{l+s+s1}{\PYZsq{}}\PY{p}{,}\PY{l+s+s1}{\PYZsq{}}\PY{l+s+s1}{rx}\PY{l+s+s1}{\PYZsq{}}\PY{p}{]}\PY{p}{)}\PY{p}{:}
        \PY{l+s+sd}{\PYZsq{}\PYZsq{}\PYZsq{}Initializer function \PYZsq{}\PYZsq{}\PYZsq{}}
        \PY{n+nb+bp}{self}\PY{o}{.}\PY{n}{n\PYZus{}layers} \PY{o}{=} \PY{n}{n\PYZus{}layers}
        \PY{n+nb+bp}{self}\PY{o}{.}\PY{n}{gate\PYZus{}set} \PY{o}{=} \PY{n}{gate\PYZus{}set}
        \PY{n+nb+bp}{self}\PY{o}{.}\PY{n}{n\PYZus{}qubits} \PY{o}{=} \PY{l+m+mi}{4}
        
        \PY{c+c1}{\PYZsh{} Initialize random parameters }
        \PY{n+nb+bp}{self}\PY{o}{.}\PY{n}{params} \PY{o}{=} \PY{n}{np}\PY{o}{.}\PY{n}{random}\PY{o}{.}\PY{n}{RandomState}\PY{p}{(}\PY{p}{)}\PY{o}{.}\PY{n}{uniform}\PY{p}{(}\PY{n}{low}\PY{o}{=}\PY{l+m+mf}{0.0}\PY{p}{,}\PY{n}{high}\PY{o}{=}\PY{l+m+mi}{2}\PY{o}{*}\PY{n}{np}\PY{o}{.}\PY{n}{pi}\PY{p}{,}\PY{n}{size}\PY{o}{=}\PY{p}{(}\PY{n}{n\PYZus{}layers}\PY{o}{*}\PY{l+m+mi}{8}\PY{p}{,}\PY{p}{)}\PY{p}{)}
        \PY{c+c1}{\PYZsh{} Obtain a Haar Random State from Qiskit, uses fixed seed so that results are consistent}
        \PY{n+nb+bp}{self}\PY{o}{.}\PY{n}{random\PYZus{}state} \PY{o}{=} \PY{n}{random\PYZus{}statevector}\PY{p}{(}\PY{l+m+mi}{16}\PY{p}{,}\PY{n}{seed}\PY{o}{=}\PY{l+m+mi}{32}\PY{p}{)}\PY{o}{.}\PY{n}{data}
    
    \PY{k}{def} \PY{n+nf}{gate}\PY{p}{(}\PY{n+nb+bp}{self}\PY{p}{,}\PY{n}{circ}\PY{p}{,}\PY{n}{gate}\PY{p}{,}\PY{n}{param}\PY{p}{,}\PY{n}{qr}\PY{p}{)}\PY{p}{:}
        \PY{l+s+sd}{\PYZsq{}\PYZsq{}\PYZsq{}Parametrized gate function.}
\PY{l+s+sd}{        Args:}
\PY{l+s+sd}{        =====}
\PY{l+s+sd}{        circ: QuantumCircuit }
\PY{l+s+sd}{            Quantum Circuit object from Qiskit}
\PY{l+s+sd}{        gate: str}
\PY{l+s+sd}{            Type of the Quantum gate}
\PY{l+s+sd}{        param: float}
\PY{l+s+sd}{            Parameter of the Quantum gate}
\PY{l+s+sd}{        qr: QuantumRegister}
\PY{l+s+sd}{            Quantum Register that we will apply the gate to}
\PY{l+s+sd}{        \PYZsq{}\PYZsq{}\PYZsq{}}
        \PY{k}{if} \PY{n}{gate}\PY{o}{==}\PY{l+s+s1}{\PYZsq{}}\PY{l+s+s1}{rz}\PY{l+s+s1}{\PYZsq{}}  \PY{p}{:} \PY{n}{circ}\PY{o}{.}\PY{n}{rz}\PY{p}{(}\PY{n}{param}\PY{p}{,}\PY{n}{qr}\PY{p}{)}
        \PY{k}{elif} \PY{n}{gate}\PY{o}{==}\PY{l+s+s1}{\PYZsq{}}\PY{l+s+s1}{ry}\PY{l+s+s1}{\PYZsq{}}\PY{p}{:} \PY{n}{circ}\PY{o}{.}\PY{n}{ry}\PY{p}{(}\PY{n}{param}\PY{p}{,}\PY{n}{qr}\PY{p}{)}
        \PY{k}{elif} \PY{n}{gate}\PY{o}{==}\PY{l+s+s1}{\PYZsq{}}\PY{l+s+s1}{rx}\PY{l+s+s1}{\PYZsq{}}\PY{p}{:} \PY{n}{circ}\PY{o}{.}\PY{n}{rx}\PY{p}{(}\PY{n}{param}\PY{p}{,}\PY{n}{qr}\PY{p}{)}
        \PY{k}{else}\PY{p}{:} \PY{k}{raise} \PY{n+ne}{ValueError}\PY{p}{(}\PY{l+s+s1}{\PYZsq{}}\PY{l+s+s1}{Instruction Not Defined.}\PY{l+s+s1}{\PYZsq{}}\PY{p}{)}
        
    \PY{k}{def} \PY{n+nf}{qc}\PY{p}{(}\PY{n+nb+bp}{self}\PY{p}{,} \PY{n}{params}\PY{p}{)}\PY{p}{:}
        \PY{l+s+sd}{\PYZsq{}\PYZsq{}\PYZsq{}Defines the the Quantum Circuit.\PYZsq{}\PYZsq{}\PYZsq{}}
        \PY{c+c1}{\PYZsh{} Setup the circuit}
        \PY{n}{qr} \PY{o}{=} \PY{n}{QuantumRegister}\PY{p}{(}\PY{n+nb+bp}{self}\PY{o}{.}\PY{n}{n\PYZus{}qubits}\PY{p}{,} \PY{l+s+s1}{\PYZsq{}}\PY{l+s+s1}{qr}\PY{l+s+s1}{\PYZsq{}}\PY{p}{)}
        \PY{n}{circ} \PY{o}{=} \PY{n}{QuantumCircuit}\PY{p}{(}\PY{n}{qr}\PY{p}{)}
                
        \PY{c+c1}{\PYZsh{} Repeats the block n\PYZus{}layers times}
        \PY{k}{for} \PY{n}{layer} \PY{o+ow}{in} \PY{n+nb}{range}\PY{p}{(}\PY{n+nb+bp}{self}\PY{o}{.}\PY{n}{n\PYZus{}layers}\PY{p}{)}\PY{p}{:}
            \PY{c+c1}{\PYZsh{} Even Block}
            \PY{k}{for} \PY{n}{idx} \PY{o+ow}{in} \PY{n+nb}{range}\PY{p}{(}\PY{n+nb+bp}{self}\PY{o}{.}\PY{n}{n\PYZus{}qubits}\PY{p}{)}\PY{p}{:}
                \PY{n+nb+bp}{self}\PY{o}{.}\PY{n}{gate}\PY{p}{(}\PY{n}{circ}\PY{p}{,}\PY{n+nb+bp}{self}\PY{o}{.}\PY{n}{gate\PYZus{}set}\PY{p}{[}\PY{l+m+mi}{0}\PY{p}{]}\PY{p}{,}\PY{n}{params}\PY{p}{[}\PY{n}{layer}\PY{o}{*}\PY{l+m+mi}{8}\PY{o}{+}\PY{n}{idx}\PY{p}{]}\PY{p}{,}\PY{n}{qr}\PY{p}{[}\PY{n}{idx}\PY{p}{]}\PY{p}{)}
            \PY{n}{circ}\PY{o}{.}\PY{n}{cz}\PY{p}{(}\PY{n}{qr}\PY{p}{[}\PY{l+m+mi}{0}\PY{p}{]}\PY{p}{,}\PY{n}{qr}\PY{p}{[}\PY{l+m+mi}{1}\PY{p}{]}\PY{p}{)}
            \PY{n}{circ}\PY{o}{.}\PY{n}{cz}\PY{p}{(}\PY{n}{qr}\PY{p}{[}\PY{l+m+mi}{0}\PY{p}{]}\PY{p}{,}\PY{n}{qr}\PY{p}{[}\PY{l+m+mi}{2}\PY{p}{]}\PY{p}{)}
            \PY{n}{circ}\PY{o}{.}\PY{n}{cz}\PY{p}{(}\PY{n}{qr}\PY{p}{[}\PY{l+m+mi}{0}\PY{p}{]}\PY{p}{,}\PY{n}{qr}\PY{p}{[}\PY{l+m+mi}{3}\PY{p}{]}\PY{p}{)}
            \PY{n}{circ}\PY{o}{.}\PY{n}{cz}\PY{p}{(}\PY{n}{qr}\PY{p}{[}\PY{l+m+mi}{1}\PY{p}{]}\PY{p}{,}\PY{n}{qr}\PY{p}{[}\PY{l+m+mi}{2}\PY{p}{]}\PY{p}{)}
            \PY{n}{circ}\PY{o}{.}\PY{n}{cz}\PY{p}{(}\PY{n}{qr}\PY{p}{[}\PY{l+m+mi}{1}\PY{p}{]}\PY{p}{,}\PY{n}{qr}\PY{p}{[}\PY{l+m+mi}{3}\PY{p}{]}\PY{p}{)}
            \PY{n}{circ}\PY{o}{.}\PY{n}{cz}\PY{p}{(}\PY{n}{qr}\PY{p}{[}\PY{l+m+mi}{2}\PY{p}{]}\PY{p}{,}\PY{n}{qr}\PY{p}{[}\PY{l+m+mi}{3}\PY{p}{]}\PY{p}{)}
            \PY{c+c1}{\PYZsh{} Odd Block}
            \PY{k}{for} \PY{n}{idx} \PY{o+ow}{in} \PY{n+nb}{range}\PY{p}{(}\PY{l+m+mi}{4}\PY{p}{)}\PY{p}{:}
                \PY{n+nb+bp}{self}\PY{o}{.}\PY{n}{gate}\PY{p}{(}\PY{n}{circ}\PY{p}{,}\PY{n+nb+bp}{self}\PY{o}{.}\PY{n}{gate\PYZus{}set}\PY{p}{[}\PY{l+m+mi}{1}\PY{p}{]}\PY{p}{,}\PY{n}{params}\PY{p}{[}\PY{n}{layer}\PY{o}{*}\PY{l+m+mi}{8}\PY{o}{+}\PY{n}{idx}\PY{o}{+}\PY{n+nb+bp}{self}\PY{o}{.}\PY{n}{n\PYZus{}qubits}\PY{p}{]}\PY{p}{,}\PY{n}{qr}\PY{p}{[}\PY{n}{idx}\PY{p}{]}\PY{p}{)}

        \PY{c+c1}{\PYZsh{} Select the StatevectorSimulator from the Aer provider}
        \PY{n}{simulator} \PY{o}{=} \PY{n}{Aer}\PY{o}{.}\PY{n}{get\PYZus{}backend}\PY{p}{(}\PY{l+s+s1}{\PYZsq{}}\PY{l+s+s1}{statevector\PYZus{}simulator}\PY{l+s+s1}{\PYZsq{}}\PY{p}{)}

        \PY{c+c1}{\PYZsh{} Execute}
        \PY{n}{result} \PY{o}{=} \PY{n}{execute}\PY{p}{(}\PY{n}{circ}\PY{p}{,} \PY{n}{simulator}\PY{p}{)}\PY{o}{.}\PY{n}{result}\PY{p}{(}\PY{p}{)}
        \PY{k}{return} \PY{n}{result}\PY{o}{.}\PY{n}{get\PYZus{}statevector}\PY{p}{(}\PY{n}{circ}\PY{p}{)}
    

    \PY{k}{def} \PY{n+nf}{cost\PYZus{}fn}\PY{p}{(}\PY{n+nb+bp}{self}\PY{p}{,} \PY{n}{params}\PY{p}{)}\PY{p}{:}
        \PY{l+s+sd}{\PYZsq{}\PYZsq{}\PYZsq{}Defines the cost function: distance between two states. }
\PY{l+s+sd}{        Args:}
\PY{l+s+sd}{        =====}
\PY{l+s+sd}{        params: float list}
\PY{l+s+sd}{            Parameters of the model.}
\PY{l+s+sd}{        Returns:}
\PY{l+s+sd}{        ========}
\PY{l+s+sd}{        float}
\PY{l+s+sd}{            Cost of the Paremetrized Quantum Circuit}
\PY{l+s+sd}{        \PYZsq{}\PYZsq{}\PYZsq{}}
        \PY{k}{return} \PY{n}{np}\PY{o}{.}\PY{n}{linalg}\PY{o}{.}\PY{n}{norm}\PY{p}{(}\PY{n+nb+bp}{self}\PY{o}{.}\PY{n}{qc}\PY{p}{(}\PY{n}{params}\PY{p}{)} \PY{o}{\PYZhy{}} \PY{n+nb+bp}{self}\PY{o}{.}\PY{n}{random\PYZus{}state}\PY{p}{,} \PY{n+nb}{ord} \PY{o}{=} \PY{l+m+mi}{2}\PY{p}{)}
               
    \PY{k}{def} \PY{n+nf}{train}\PY{p}{(}\PY{n+nb+bp}{self}\PY{p}{)}\PY{p}{:}
        \PY{l+s+sd}{\PYZsq{}\PYZsq{}\PYZsq{}Trainer of the circuit.}
\PY{l+s+sd}{        }
\PY{l+s+sd}{        Returns:}
\PY{l+s+sd}{        =======}
\PY{l+s+sd}{        float}
\PY{l+s+sd}{            Final cost of the model}
\PY{l+s+sd}{        \PYZsq{}\PYZsq{}\PYZsq{}}
        \PY{k}{return} \PY{n}{minimize}\PY{p}{(}\PY{n+nb+bp}{self}\PY{o}{.}\PY{n}{cost\PYZus{}fn}\PY{p}{,} \PY{n+nb+bp}{self}\PY{o}{.}\PY{n}{params}\PY{p}{,} \PY{n}{method}\PY{o}{=}\PY{l+s+s1}{\PYZsq{}}\PY{l+s+s1}{COBYLA}\PY{l+s+s1}{\PYZsq{}}\PY{p}{)}\PY{o}{.}\PY{n}{fun}
\end{Verbatim}
\end{tcolorbox}

    We are ready to define how many layers we want to test. Here, it is also
good idea to run the same model several times and take an average, as
Quantum Circuits sometimes do not train very well due to a bad
initialization. Since we randomly initialize the circuits every time, we
will run each model 3 times and use averaged results.

    \begin{tcolorbox}[breakable, size=fbox, boxrule=1pt, pad at break*=1mm,colback=cellbackground, colframe=cellborder]
\prompt{In}{incolor}{3}{\boxspacing}
\begin{Verbatim}[commandchars=\\\{\}]
\PY{c+c1}{\PYZsh{} Sets number of layers to be tested.}
\PY{n}{n\PYZus{}layers} \PY{o}{=} \PY{l+m+mi}{20}
\PY{c+c1}{\PYZsh{} A list of number of layers. }
\PY{n}{layer\PYZus{}list} \PY{o}{=} \PY{n}{np}\PY{o}{.}\PY{n}{linspace}\PY{p}{(}\PY{l+m+mi}{1}\PY{p}{,}\PY{n}{n\PYZus{}layers}\PY{p}{,}\PY{n}{n\PYZus{}layers}\PY{p}{,}\PY{n}{dtype}\PY{o}{=}\PY{n+nb}{int}\PY{p}{)}
\PY{n}{n\PYZus{}runs} \PY{o}{=} \PY{l+m+mi}{3}
\PY{c+c1}{\PYZsh{} Arrays to store the costs for each model.}
\PY{n}{costs} \PY{o}{=} \PY{p}{[}\PY{p}{]}
\PY{n}{errors} \PY{o}{=} \PY{p}{[}\PY{p}{]}
\end{Verbatim}
\end{tcolorbox}

    Let's train the model up to 20 layers and log the \(\varepsilon\).

    \begin{tcolorbox}[breakable, size=fbox, boxrule=1pt, pad at break*=1mm,colback=cellbackground, colframe=cellborder]
\prompt{In}{incolor}{4}{\boxspacing}
\begin{Verbatim}[commandchars=\\\{\}]
\PY{c+c1}{\PYZsh{} Starts a loop where the model is independently trained up to n\PYZus{}layer layers.}
\PY{c+c1}{\PYZsh{} all models are independently trained n\PYZus{}runs and the average of the loss curves are recorded.}
\PY{k}{for} \PY{n}{idx}\PY{p}{,} \PY{n}{n\PYZus{}layer} \PY{o+ow}{in} \PY{n+nb}{enumerate}\PY{p}{(}\PY{n}{layer\PYZus{}list}\PY{p}{)}\PY{p}{:}
    \PY{c+c1}{\PYZsh{} start timer}
    \PY{n}{t0} \PY{o}{=} \PY{n}{time}\PY{o}{.}\PY{n}{time}\PY{p}{(}\PY{p}{)} 
    \PY{n}{run\PYZus{}costs} \PY{o}{=} \PY{n}{np}\PY{o}{.}\PY{n}{zeros}\PY{p}{(}\PY{n}{n\PYZus{}runs}\PY{p}{)}
    \PY{c+c1}{\PYZsh{} Train the model and append the costs to run\PYZus{}cost list.}
    \PY{k}{for} \PY{n}{i} \PY{o+ow}{in} \PY{n+nb}{range}\PY{p}{(}\PY{n}{n\PYZus{}runs}\PY{p}{)}\PY{p}{:}
        \PY{c+c1}{\PYZsh{} Instantiate the QNN model with the given parameters.}
        \PY{n}{model} \PY{o}{=} \PY{n}{QNN}\PY{p}{(}\PY{n}{n\PYZus{}layers}\PY{o}{=}\PY{n}{n\PYZus{}layer}\PY{p}{)}
        \PY{n}{run\PYZus{}costs}\PY{p}{[}\PY{n}{i}\PY{p}{]} \PY{o}{=} \PY{n}{model}\PY{o}{.}\PY{n}{train}\PY{p}{(}\PY{p}{)}
    \PY{n}{costs}\PY{o}{.}\PY{n}{append}\PY{p}{(}\PY{n}{np}\PY{o}{.}\PY{n}{mean}\PY{p}{(}\PY{n}{run\PYZus{}costs}\PY{p}{)}\PY{p}{)}
    \PY{n}{errors}\PY{o}{.}\PY{n}{append}\PY{p}{(}\PY{n}{np}\PY{o}{.}\PY{n}{std}\PY{p}{(}\PY{n}{run\PYZus{}costs}\PY{p}{)}\PY{p}{)}
    \PY{c+c1}{\PYZsh{} end timer and record duration}
    \PY{n}{duration} \PY{o}{=} \PY{n}{time}\PY{o}{.}\PY{n}{time}\PY{p}{(}\PY{p}{)} \PY{o}{\PYZhy{}} \PY{n}{t0} 
    \PY{c+c1}{\PYZsh{} calculate the mean and std. for multiple runs}
    \PY{n+nb}{print}\PY{p}{(}\PY{l+s+s1}{\PYZsq{}}\PY{l+s+s1}{Training completed for }\PY{l+s+si}{\PYZob{}:d\PYZcb{}}\PY{l+s+s1}{ layers. Final Cost: }\PY{l+s+si}{\PYZob{}:1.3f\PYZcb{}}\PY{l+s+s1}{ +/\PYZhy{} }\PY{l+s+si}{\PYZob{}:1.3f\PYZcb{}}\PY{l+s+s1}{ in }\PY{l+s+si}{\PYZob{}:2.0f\PYZcb{}}\PY{l+s+s1}{m}\PY{l+s+si}{\PYZob{}:2.0f\PYZcb{}}\PY{l+s+s1}{s}\PY{l+s+s1}{\PYZsq{}}\PY{o}{.}\PY{n}{format}\PY{p}{(}\PY{n}{n\PYZus{}layer}\PY{p}{,} \PY{n}{costs}\PY{p}{[}\PY{n}{idx}\PY{p}{]}\PY{p}{,} \PY{n}{errors}\PY{p}{[}\PY{n}{idx}\PY{p}{]}\PY{p}{,} \PY{n}{duration}\PY{o}{/}\PY{o}{/}\PY{l+m+mi}{60}\PY{p}{,} \PY{n}{duration}\PY{o}{\PYZpc{}}\PY{k}{60}))
\end{Verbatim}
\end{tcolorbox}

    \begin{Verbatim}[commandchars=\\\{\}]
Training completed for 1 layers. Final Cost: 1.052 +/- 0.029 in  0m 2s
Training completed for 2 layers. Final Cost: 0.474 +/- 0.000 in  0m14s
Training completed for 3 layers. Final Cost: 0.356 +/- 0.127 in  0m26s
Training completed for 4 layers. Final Cost: 0.330 +/- 0.045 in  0m31s
Training completed for 5 layers. Final Cost: 0.196 +/- 0.082 in  0m36s
Training completed for 6 layers. Final Cost: 0.163 +/- 0.018 in  0m38s
Training completed for 7 layers. Final Cost: 0.099 +/- 0.036 in  0m44s
Training completed for 8 layers. Final Cost: 0.137 +/- 0.044 in  0m49s
Training completed for 9 layers. Final Cost: 0.156 +/- 0.064 in  0m54s
Training completed for 10 layers. Final Cost: 0.134 +/- 0.051 in  1m 0s
Training completed for 11 layers. Final Cost: 0.103 +/- 0.038 in  1m 4s
Training completed for 12 layers. Final Cost: 0.136 +/- 0.021 in  1m15s
Training completed for 13 layers. Final Cost: 0.127 +/- 0.040 in  1m 8s
Training completed for 14 layers. Final Cost: 0.118 +/- 0.043 in  1m14s
Training completed for 15 layers. Final Cost: 0.145 +/- 0.051 in  1m19s
Training completed for 16 layers. Final Cost: 0.123 +/- 0.020 in  1m26s
Training completed for 17 layers. Final Cost: 0.204 +/- 0.086 in  1m48s
Training completed for 18 layers. Final Cost: 0.107 +/- 0.042 in  2m 1s
Training completed for 19 layers. Final Cost: 0.106 +/- 0.002 in  2m 5s
Training completed for 20 layers. Final Cost: 0.148 +/- 0.024 in  2m11s
    \end{Verbatim}

    Plot \(\varepsilon\) vs.~\(N_{layers}\)

    \begin{tcolorbox}[breakable, size=fbox, boxrule=1pt, pad at break*=1mm,colback=cellbackground, colframe=cellborder]
\prompt{In}{incolor}{ }{\boxspacing}
\begin{Verbatim}[commandchars=\\\{\}]
\PY{c+c1}{\PYZsh{} define plot area}
\PY{n}{fig}\PY{p}{,} \PY{n}{ax} \PY{o}{=} \PY{n}{plt}\PY{o}{.}\PY{n}{subplots}\PY{p}{(}\PY{l+m+mi}{1}\PY{p}{,}\PY{n}{figsize}\PY{o}{=}\PY{p}{(}\PY{l+m+mi}{5}\PY{p}{,} \PY{l+m+mi}{2}\PY{p}{)}\PY{p}{)}
\PY{c+c1}{\PYZsh{} set xticks}
\PY{n}{plt}\PY{o}{.}\PY{n}{xticks}\PY{p}{(}\PY{p}{[}\PY{n}{i}\PY{o}{*}\PY{p}{(}\PY{n}{n\PYZus{}layers}\PY{o}{/}\PY{o}{/}\PY{l+m+mi}{5}\PY{p}{)} \PY{k}{for} \PY{n}{i} \PY{o+ow}{in} \PY{n+nb}{range}\PY{p}{(}\PY{l+m+mi}{2}\PY{o}{+}\PY{p}{(}\PY{n}{n\PYZus{}layers}\PY{o}{/}\PY{o}{/}\PY{l+m+mi}{5}\PY{p}{)}\PY{p}{)}\PY{p}{]}\PY{p}{)}
\PY{c+c1}{\PYZsh{} plot with error bars}
\PY{n}{plt}\PY{o}{.}\PY{n}{errorbar}\PY{p}{(}\PY{n}{x}\PY{o}{=}\PY{n}{layer\PYZus{}list}\PY{p}{,} \PY{n}{y}\PY{o}{=}\PY{n}{costs}\PY{p}{,} \PY{n}{yerr}\PY{o}{=}\PY{n}{errors}\PY{p}{)}
\PY{c+c1}{\PYZsh{} set labels}
\PY{n}{ax}\PY{o}{.}\PY{n}{set\PYZus{}xlabel}\PY{p}{(}\PY{l+s+sa}{r}\PY{l+s+s1}{\PYZsq{}}\PY{l+s+s1}{\PYZdl{}N\PYZus{}}\PY{l+s+si}{\PYZob{}layers\PYZcb{}}\PY{l+s+s1}{\PYZdl{}}\PY{l+s+s1}{\PYZsq{}}\PY{p}{)}
\PY{n}{ax}\PY{o}{.}\PY{n}{set\PYZus{}ylabel}\PY{p}{(}\PY{l+s+sa}{r}\PY{l+s+s1}{\PYZsq{}}\PY{l+s+s1}{\PYZdl{}}\PY{l+s+s1}{\PYZbs{}}\PY{l+s+s1}{varepsilon\PYZdl{}}\PY{l+s+s1}{\PYZsq{}}\PY{p}{)}
\PY{c+c1}{\PYZsh{} show plot}
\PY{n}{plt}\PY{o}{.}\PY{n}{show}\PY{p}{(}\PY{p}{)}
\end{Verbatim}
\end{tcolorbox}

    \begin{center}
        \adjustimage{width=\linewidth}{output_10_0.png}

    \end{center}
    { \hspace*{\fill} \\}
    
    Above plot shows us that, after some \(N_{layers}\) the model stops
improving. The main reason for this is that the generalizability of the
Quantum Circuits saturates after some layers {[}1{]}. Also, it gets
harder to train Quantum Circuits as the depth increases {[}2,3{]}. As a
result, we obtain a convergence plateau.

In this question, we try to show the effect of increasing the depth of
the Quantum Circuits. Although, there are methods {[}3,4{]} to obtain
better \(\varepsilon\), the aim is not to get the best possible result.

    \hypertarget{solution-to-bonus-question}{%
\section{Solution to Bonus Question}\label{solution-to-bonus-question}}

The bonus question asks to vary the gates of the model. We define a set
of gate combinations and test them using the setting \(N_{layer}\) = 9.
Here, we only use the \(R_X\), \(R_Y\) and \(R_Z\) gates but this
example can be extended with the use of other parametrized gates such as
\(U_1\), \(U_2\), \(U_3\), etc.

    \begin{tcolorbox}[breakable, size=fbox, boxrule=1pt, pad at break*=1mm,colback=cellbackground, colframe=cellborder]
\prompt{In}{incolor}{6}{\boxspacing}
\begin{Verbatim}[commandchars=\\\{\}]
\PY{c+c1}{\PYZsh{} Choose a layer}
\PY{n}{n\PYZus{}layer} \PY{o}{=} \PY{l+m+mi}{9}
\PY{n}{n\PYZus{}runs} \PY{o}{=} \PY{l+m+mi}{3}
\PY{c+c1}{\PYZsh{} Creates a set of gates to run}
\PY{n}{gate\PYZus{}list} \PY{o}{=} \PY{p}{[}
    \PY{p}{[}\PY{l+s+s1}{\PYZsq{}}\PY{l+s+s1}{rz}\PY{l+s+s1}{\PYZsq{}}\PY{p}{,}\PY{l+s+s1}{\PYZsq{}}\PY{l+s+s1}{rz}\PY{l+s+s1}{\PYZsq{}}\PY{p}{]}\PY{p}{,}
    \PY{p}{[}\PY{l+s+s1}{\PYZsq{}}\PY{l+s+s1}{rz}\PY{l+s+s1}{\PYZsq{}}\PY{p}{,}\PY{l+s+s1}{\PYZsq{}}\PY{l+s+s1}{ry}\PY{l+s+s1}{\PYZsq{}}\PY{p}{]}\PY{p}{,}
    \PY{p}{[}\PY{l+s+s1}{\PYZsq{}}\PY{l+s+s1}{rz}\PY{l+s+s1}{\PYZsq{}}\PY{p}{,}\PY{l+s+s1}{\PYZsq{}}\PY{l+s+s1}{rx}\PY{l+s+s1}{\PYZsq{}}\PY{p}{]}\PY{p}{,}
    \PY{p}{[}\PY{l+s+s1}{\PYZsq{}}\PY{l+s+s1}{ry}\PY{l+s+s1}{\PYZsq{}}\PY{p}{,}\PY{l+s+s1}{\PYZsq{}}\PY{l+s+s1}{rz}\PY{l+s+s1}{\PYZsq{}}\PY{p}{]}\PY{p}{,}
    \PY{p}{[}\PY{l+s+s1}{\PYZsq{}}\PY{l+s+s1}{ry}\PY{l+s+s1}{\PYZsq{}}\PY{p}{,}\PY{l+s+s1}{\PYZsq{}}\PY{l+s+s1}{ry}\PY{l+s+s1}{\PYZsq{}}\PY{p}{]}\PY{p}{,}
    \PY{p}{[}\PY{l+s+s1}{\PYZsq{}}\PY{l+s+s1}{ry}\PY{l+s+s1}{\PYZsq{}}\PY{p}{,}\PY{l+s+s1}{\PYZsq{}}\PY{l+s+s1}{rx}\PY{l+s+s1}{\PYZsq{}}\PY{p}{]}\PY{p}{,}
    \PY{p}{[}\PY{l+s+s1}{\PYZsq{}}\PY{l+s+s1}{rx}\PY{l+s+s1}{\PYZsq{}}\PY{p}{,}\PY{l+s+s1}{\PYZsq{}}\PY{l+s+s1}{rz}\PY{l+s+s1}{\PYZsq{}}\PY{p}{]}\PY{p}{,}
    \PY{p}{[}\PY{l+s+s1}{\PYZsq{}}\PY{l+s+s1}{rx}\PY{l+s+s1}{\PYZsq{}}\PY{p}{,}\PY{l+s+s1}{\PYZsq{}}\PY{l+s+s1}{ry}\PY{l+s+s1}{\PYZsq{}}\PY{p}{]}\PY{p}{,}
    \PY{p}{[}\PY{l+s+s1}{\PYZsq{}}\PY{l+s+s1}{rx}\PY{l+s+s1}{\PYZsq{}}\PY{p}{,}\PY{l+s+s1}{\PYZsq{}}\PY{l+s+s1}{rx}\PY{l+s+s1}{\PYZsq{}}\PY{p}{]}\PY{p}{,}
    \PY{p}{]}
\PY{n}{n\PYZus{}gate\PYZus{}sets} \PY{o}{=} \PY{n+nb}{len}\PY{p}{(}\PY{n}{gate\PYZus{}list}\PY{p}{)}
\PY{c+c1}{\PYZsh{} Arrays to store the costs for each model.}
\PY{n}{costs} \PY{o}{=} \PY{p}{[}\PY{p}{]}
\PY{n}{errors} \PY{o}{=} \PY{p}{[}\PY{p}{]}
\end{Verbatim}
\end{tcolorbox}

    \begin{tcolorbox}[breakable, size=fbox, boxrule=1pt, pad at break*=1mm,colback=cellbackground, colframe=cellborder]
\prompt{In}{incolor}{7}{\boxspacing}
\begin{Verbatim}[commandchars=\\\{\}]
\PY{c+c1}{\PYZsh{} Training loop}
\PY{k}{for} \PY{n}{idx}\PY{p}{,} \PY{n}{gates} \PY{o+ow}{in} \PY{n+nb}{enumerate}\PY{p}{(}\PY{n}{gate\PYZus{}list}\PY{p}{)}\PY{p}{:}
    \PY{c+c1}{\PYZsh{} start timer}
    \PY{n}{t0} \PY{o}{=} \PY{n}{time}\PY{o}{.}\PY{n}{time}\PY{p}{(}\PY{p}{)} 
    \PY{n}{run\PYZus{}costs} \PY{o}{=} \PY{n}{np}\PY{o}{.}\PY{n}{zeros}\PY{p}{(}\PY{n}{n\PYZus{}runs}\PY{p}{)}
    \PY{k}{for} \PY{n}{run} \PY{o+ow}{in} \PY{n+nb}{range}\PY{p}{(}\PY{n}{n\PYZus{}runs}\PY{p}{)}\PY{p}{:}
        \PY{n}{model} \PY{o}{=} \PY{n}{QNN}\PY{p}{(}\PY{n}{n\PYZus{}layers}\PY{o}{=}\PY{n}{n\PYZus{}layer}\PY{p}{,}\PY{n}{gate\PYZus{}set}\PY{o}{=}\PY{n}{gates}\PY{p}{)}
        \PY{n}{run\PYZus{}costs}\PY{p}{[}\PY{n}{run}\PY{p}{]} \PY{o}{=} \PY{n}{model}\PY{o}{.}\PY{n}{train}\PY{p}{(}\PY{p}{)}
        
    \PY{n}{costs}\PY{o}{.}\PY{n}{append}\PY{p}{(}\PY{n}{np}\PY{o}{.}\PY{n}{mean}\PY{p}{(}\PY{n}{run\PYZus{}costs}\PY{p}{)}\PY{p}{)}
    \PY{n}{errors}\PY{o}{.}\PY{n}{append}\PY{p}{(}\PY{n}{np}\PY{o}{.}\PY{n}{std}\PY{p}{(}\PY{n}{run\PYZus{}costs}\PY{p}{)}\PY{p}{)}
    \PY{c+c1}{\PYZsh{} end timer and record duration}
    \PY{n}{duration} \PY{o}{=} \PY{n}{time}\PY{o}{.}\PY{n}{time}\PY{p}{(}\PY{p}{)} \PY{o}{\PYZhy{}} \PY{n}{t0} 
    \PY{c+c1}{\PYZsh{} calculate the mean and std. for multiple runs}
    \PY{n+nb}{print}\PY{p}{(}\PY{l+s+s1}{\PYZsq{}}\PY{l+s+s1}{Training completed for even gates: }\PY{l+s+si}{\PYZob{}\PYZcb{}}\PY{l+s+s1}{, odd gates: }\PY{l+s+si}{\PYZob{}\PYZcb{}}\PY{l+s+s1}{ layers. Final Cost: }\PY{l+s+si}{\PYZob{}:1.3f\PYZcb{}}\PY{l+s+s1}{ +/\PYZhy{} }\PY{l+s+si}{\PYZob{}:1.3f\PYZcb{}}\PY{l+s+s1}{ in }\PY{l+s+si}{\PYZob{}:2.0f\PYZcb{}}\PY{l+s+s1}{m}\PY{l+s+si}{\PYZob{}:2.0f\PYZcb{}}\PY{l+s+s1}{s}\PY{l+s+s1}{\PYZsq{}}\PY{o}{.}\PY{n}{format}\PY{p}{(}\PY{n}{gates}\PY{p}{[}\PY{l+m+mi}{0}\PY{p}{]}\PY{p}{,}\PY{n}{gates}\PY{p}{[}\PY{l+m+mi}{1}\PY{p}{]}\PY{p}{,} \PY{n}{costs}\PY{p}{[}\PY{n}{idx}\PY{p}{]}\PY{p}{,} \PY{n}{errors}\PY{p}{[}\PY{n}{idx}\PY{p}{]}\PY{p}{,} \PY{n}{duration}\PY{o}{/}\PY{o}{/}\PY{l+m+mi}{60}\PY{p}{,} \PY{n}{duration}\PY{o}{\PYZpc{}}\PY{k}{60}))
\end{Verbatim}
\end{tcolorbox}

    \begin{Verbatim}[commandchars=\\\{\}]
Training completed for even gates: rz, odd gates: rz layers. Final Cost: 1.649
+/- 0.000 in  0m23s
Training completed for even gates: rz, odd gates: ry layers. Final Cost: 0.128
+/- 0.052 in  0m58s
Training completed for even gates: rz, odd gates: rx layers. Final Cost: 0.101
+/- 0.032 in  0m59s
Training completed for even gates: ry, odd gates: rz layers. Final Cost: 0.127
+/- 0.041 in  0m57s
Training completed for even gates: ry, odd gates: ry layers. Final Cost: 0.710
+/- 0.000 in  1m22s
Training completed for even gates: ry, odd gates: rx layers. Final Cost: 0.153
+/- 0.062 in  1m22s
Training completed for even gates: rx, odd gates: rz layers. Final Cost: 0.072
+/- 0.032 in  0m57s
Training completed for even gates: rx, odd gates: ry layers. Final Cost: 0.218
+/- 0.062 in  1m24s
Training completed for even gates: rx, odd gates: rx layers. Final Cost: 0.820
+/- 0.000 in  1m22s
    \end{Verbatim}

    Plot the final \(\varepsilon\) of each model against the gate sets.

    \begin{tcolorbox}[breakable, size=fbox, boxrule=1pt, pad at break*=1mm,colback=cellbackground, colframe=cellborder]
\prompt{In}{incolor}{ }{\boxspacing}
\begin{Verbatim}[commandchars=\\\{\}]
\PY{n}{fig}\PY{p}{,} \PY{n}{ax} \PY{o}{=} \PY{n}{plt}\PY{o}{.}\PY{n}{subplots}\PY{p}{(}\PY{l+m+mi}{1}\PY{p}{,} \PY{n}{figsize}\PY{o}{=}\PY{p}{(}\PY{l+m+mi}{7}\PY{p}{,} \PY{l+m+mi}{2}\PY{p}{)}\PY{p}{)}
\PY{n}{x\PYZus{}ticks} \PY{o}{=} \PY{p}{[}\PY{l+s+s1}{\PYZsq{}}\PY{l+s+s1}{\PYZob{}}\PY{l+s+s1}{\PYZsq{}}\PY{o}{+}\PY{n}{gate\PYZus{}list}\PY{p}{[}\PY{n}{i}\PY{p}{]}\PY{p}{[}\PY{l+m+mi}{0}\PY{p}{]}\PY{o}{+}\PY{l+s+s1}{\PYZsq{}}\PY{l+s+s1}{, }\PY{l+s+s1}{\PYZsq{}}\PY{o}{+}\PY{n}{gate\PYZus{}list}\PY{p}{[}\PY{n}{i}\PY{p}{]}\PY{p}{[}\PY{l+m+mi}{1}\PY{p}{]}\PY{o}{+}\PY{l+s+s1}{\PYZsq{}}\PY{l+s+s1}{\PYZcb{}}\PY{l+s+s1}{\PYZsq{}} \PY{k}{for} \PY{n}{i} \PY{o+ow}{in} \PY{n+nb}{range}\PY{p}{(}\PY{n}{n\PYZus{}gate\PYZus{}sets}\PY{p}{)}\PY{p}{]}
\PY{n}{plt}\PY{o}{.}\PY{n}{xticks}\PY{p}{(}\PY{n+nb}{range}\PY{p}{(}\PY{n}{n\PYZus{}gate\PYZus{}sets}\PY{p}{)}\PY{p}{,} \PY{n}{x\PYZus{}ticks}\PY{p}{)}
\PY{n}{plt}\PY{o}{.}\PY{n}{plot}\PY{p}{(}\PY{n+nb}{range}\PY{p}{(}\PY{n}{n\PYZus{}gate\PYZus{}sets}\PY{p}{)}\PY{p}{,} \PY{n}{costs}\PY{p}{,} \PY{n}{marker}\PY{o}{=}\PY{l+s+s2}{\PYZdq{}}\PY{l+s+s2}{d}\PY{l+s+s2}{\PYZdq{}}\PY{p}{,} \PY{n}{linestyle}\PY{o}{=}\PY{l+s+s2}{\PYZdq{}}\PY{l+s+s2}{None}\PY{l+s+s2}{\PYZdq{}}\PY{p}{)}
\PY{n}{ax}\PY{o}{.}\PY{n}{set\PYZus{}ylabel}\PY{p}{(}\PY{l+s+sa}{r}\PY{l+s+s1}{\PYZsq{}}\PY{l+s+s1}{\PYZdl{}}\PY{l+s+s1}{\PYZbs{}}\PY{l+s+s1}{varepsilon\PYZdl{}}\PY{l+s+s1}{\PYZsq{}}\PY{p}{)}
\PY{n}{plt}\PY{o}{.}\PY{n}{show}\PY{p}{(}\PY{p}{)}
\end{Verbatim}
\end{tcolorbox}

    \begin{center}
    \adjustimage{width=\linewidth}{output_16_0.png}
    \end{center}
    { \hspace*{\fill} \\}
    
    This plot shows us that we get the best results when we have
combinations of different gate types. The perfomance decreases
significantly, when we have the same gate types for even and odd blocks.
However, there is a much wore case, which is the case where there are
only \(R_Z\) gates.

The reason for getting the worse \(\varepsilon\) in the only \(R_Z\)
case is that the qubits being initialized in the
\(\newcommand{\ket}[1]{\left|{#1}\right\rangle} \ket{z;0}\) state. When
we use only the R\_Z gate, we can't get out of the
\(\newcommand{\ket}[1]{\left|{#1}\right\rangle} \ket{0}\) state.

This situation also extends to the cases when we only use \(R_X\) or
\(R_Y\) gates. In these cases, the sytem can get out of the
\(\newcommand{\ket}[1]{\left|{#1}\right\rangle} \ket{0}\) state.
However, since there is only 1 degree of freedom, the model can't use
the full potential of the bloch sphere.

The models with different gates performs the best as we can make us of
complete bloch sphere. This result can also be inferred from linear
algebra, where a general \(U_3\) unitary transformation can be
decomposed to a set of rotations in different axes.

    \hypertarget{references}{%
\section{References}\label{references}}

{[}1{]} S. Sim, P. D. Johnson, and A. Aspuru‐Guzik, ``Expressibility and
Entangling Capability of Parameterized Quantum Circuits for Hybrid
Quantum‐Classical Algorithms,'' Adv. Quantum Technol., vol.~2, no. 12,
p.~1900070, 2019, doi: 10.1002/qute.201900070. Available:
https://arxiv.org/abs/1905.10876

{[}2{]} J. R. McClean, S. Boixo, V. N. Smelyanskiy, R. Babbush, and H.
Neven, ``Barren plateaus in quantum neural network training
landscapes,'' Nat. Commun., vol.~9, no. 1, pp.~1--6, 2018, doi:
10.1038/s41467-018-07090-4. Available: https://arxiv.org/abs/1803.11173

{[}3{]} E. Grant, L. Wossnig, M. Ostaszewski, and M. Benedetti, ``An
initialization strategy for addressing barren plateaus in parametrized
quantum circuits,'' Quantum, vol.~3, p.~214, 2019, doi:
10.22331/q-2019-12-09-214. Available: https://arxiv.org/abs/1903.05076

{[}4{]} A. Skolik, J. R. McClean, M. Mohseni, P. van der Smagt, and M.
Leib, ``Layerwise learning for quantum neural networks,'' 2020,
{[}Online{]}. Available: http://arxiv.org/abs/2006.14904.


\section*{End Note}

This document is partially auto-generated and originally written as a jupyter notebook. There might be some discrepancies in means of formatting. Please alse refer to \url{https://github.com/cnktysz/qosf-screening-tasks/blob/master/task1/Task1.ipynb}

    
\end{document}
